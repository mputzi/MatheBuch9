

\documentclass{standalone}
\usepackage[T1]{fontenc}
\usepackage[utf8]{inputenc}
\usepackage[ngerman]{babel}
\usepackage{pst-all}
\usepackage{pst-3dplot}
\usepackage{pst-solides3d}

% \pagestyle{empty}
% \headheight=0pt
% \headsep=0pt
% \paperwidth=14cm
% \paperheight=10cm
% \topmargin=0cm
% \oddsidemargin=-2cm
% \evensidemargin=0cm
% \parindent=0sp
% \special{papersize=\the\paperwidth,\the\paperheight}
%opening
\title{}
\author{}



\begin{document}
  \begin{pspicture}(-5,-1)(6,3.7)
% \psset{Beta=20,Alpha=160,zMax=2, zMin=-2, yMax=2, yMin=-2, xMin=-0.5, xMax=6.5, pOrigin=c, unit=1cm}
%  \pstThreeDCoor[]
%  \psset{linewidth=0.5pt}
  \psset{viewpoint=100 0 20 rtp2xyz,Decran=100}
%   \psset{lightsrc=10 20 30}
  \psset{lightsrc=viewpoint}
  \psset{solidmemory}
  
% \axesIIID[](0,0,0)(1,1,1)  

 \psPoint(0,-1,0){A}
 \psPoint(0,-1,3){B}
%  \uput[r](A){$\mathrm{A}$}
%  \uput[r](B){$\mathrm{B}$}
 \pcline[offset=-0.2]{|<->|}(A)(B)
 \nbput*{$h$}
 
 \psSolid[axe=0 0 1,base=
-1 -3 1 -3 0 -1,
object=prisme,
action=draw*,
linecolor=black,
opacity=0.5,
fillcolor=gray!20,
fcol=0 (red),
show=all,
 h=3, name=prisma]
 
 \psset{fontsize=40}
 \psSolid[object=plan,action=none,opacity=0.5, definition=solidface,args=prisma 0,name=P0,]
 \psSolid[object=point, action=none, definition=solidcentreface, args=prisma 1, name=G]
 \psProjection[object=texte,linecolor=black,text=G,plan=P0,]
%  \psset{phi=135}

  \psSolid[object=plan, action=none, definition=solidface, args=prisma 0, name=P1] (G)
  \psProjection[object=texte,linecolor=gray,text={G},plan=P1]
 
  \psSolid[axe=0 0.5 1,base=
1 3 -1 3 0 1,
object=prisme,
action=draw*,
linecolor=black,
opacity=0.5,
fillcolor=gray!20,
fcol=0 (red),
show=all,
 h=3.354, name=prismatwo]
%   \psSolid[object=plan, action=draw, definition=solidface, args=prismatwo 0,name=P2] 
 \psSolid[object=plan, action=draw, definition=normalpoint, args={0 0 0 [1 0 0 0 0 1]}, fillcolor=white,
base=-2 2 -4 5,name=P2] 
 \psSolid[object=plan,  definition=normalpoint, args={0 0 3 [1 0 0 0 0 1]},
base=-2 2 -4 5,visibility=false, linestyle=dashed,action=draw,name=P3]

%  \psPoint(0,0,3.5){G} \rput(G){$G$}
%  \psSolid[object=line,
%     args= 0 0 3.5 0 0 3]
 
 
 
%  \pstIIIDCylinder[fillcolor=gray,fillstyle=solid,linecolor=black!20, hiddenLine=true,increment=0.4](-3,0,0){1.41}{0.6}
%  \pstPlanePut[plane=xy,pOrigin=c](-3,0,1){$A= 2\pi$}
%  \pstIIIDCylinder[hiddenLine=true, increment=180](-3,0,0){1.41}{1}
% 
%  \pstThreeDLine[linecolor=blue](-1.59,0,1)(0,0,1)
%  \pstThreeDLine[linecolor=blue](-1.59,0,0.6)(0,0,0.6)
%  \pstThreeDLine[linecolor=blue](-1.59,0,0)(0,0,0)
%  
%  \pstIIIDCylinder[RotY=90, hiddenLine=true, increment=180](0,0,0){1}{1}
%  \pstIIIDCylinder[RotY=90, fillcolor=black!20,fillstyle=solid,linecolor=gray,increment=2](0,0,0){0.6}{1.666}
% %  \pstThreeDCircle[fillcolor=black!20,fillstyle=solid](0,0,0)(0,0,0.6)(0,1,0)
% 
%  \pstThreeDLine[linecolor=red](1.666,0,0)(3,0,0)
%  
%   \parametricplotThreeD[plotstyle=curve,drawStyle=xLines,xPlotpoints=25,yPlotpoints=25,hiddenLine=true](0,360)(1.666,6){u 1 u div t sin mul 1 u div t cos mul}
%  \psplotThreeD[plotstyle=curve](1,6)(0,0){1 x div}
%  
%  \pstThreeDLine[linestyle=dashed](-3,0,1) (-3,0,1.8)
%  \pstThreeDLine[linestyle=dashed](-1.59,0,1) (-1.59,0,1.8)
%  \pstThreeDLine[](-1.59,0,1.8)(-3,0,1.8)
%  \pstPlanePut[plane=xz](-2.25,0,2){$\sqrt{2}$}
%  
%  \pstThreeDLine[](-1,0,0)(-1,0,1)
%  \pstPlanePut[plane=xz](-1,0,0.5){$1$}
%  
%  \pstThreeDLine[](0,0,1.8)(1,0,1.8)
%  \pstThreeDLine[linestyle=dashed](1,0,1) (1,0,1.8)
%  \pstThreeDLine[linestyle=dashed](0,0,1) (0,0,1.8)
%  \pstPlanePut[plane=xz](0.5,0,2){$1$}
%  
%  \pstPlanePut[plane=xz](3,0,1){$r=\frac{1}{x}$}
%  
%  \pstPlanePut[plane=xz,pOrigin=c](0.8,0,-1.5){$M=2\pi$}
\end{pspicture}


\end{document}
